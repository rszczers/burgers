\chapter{Ustalenia początkowe}
\begin{definicja}\cite[s. 253]{maurin2}
    Niech M będzie przestrzeń Hausdorffa lokalnie homeomorficzna z
    ustaloną przestrzenią Banacha \(E\). Wówczas przestrzeń \(M\) nazywamy \emph{rozmaitością
    topologiczną}
    
    Niech \(M\) będzie lokalne homeomorfizmy
    \(M \supset \Omega_{\chi} \rightarrow \chi(\Omega_{\chi}) \subset E\). \(M\) nazywamy
\emph{mapą}.

\begin{definicja}\cite[s. 41]{maurin2}
    Niech \((X, \tau)\) będzie przestrzenią Hausdorffa, przy czym \(X\) jest grupą z działaniem
    grupowym \(X\times X \ni (x, y) \rightarrow x\dot y \in X\) i elementem neutralnym \(e\).
    Przestrzeń \((X, \tau)\) nazywa się \emph{grupą topologiczną}, gdy odwzorowania:
    \begin{align*}
        X\times X \ni (x, y) \rightarrow x \dot y \in Y, \\ 
        X \ni x \rightarrow x^{-1} \in X
    \end{align*}
\end{definicja} są ciągłe.

\end{definicja}
\begin{definicja}\cite[s. 253]{maurin2}
\emph{Strukturą różniczkowalną} (albo też \emph{atlasem}) klasy \(p\) nazywa się taką
rodzinę \( \{(\chi, \Omega_{\chi})\} \) map, że:
\begin{enumerate}[label=(\alph*)]
    \item Rodzina \(\{\Omega_{\chi}\}\) stanowi otwarte pokrycie przestrzeni \(M\) (tzn. każdy punkt
        \(x\in M\) ma otoczenie homeomorficzne z podzbiorem w \(E\)).
    \item Jeśli \((\chi_1, \Omega_{\chi_1})\) i \((\chi_2, \Omega_{\chi_2})\), to odwzorowanie
        \begin{equation*}
            (\chi_1 \circ \chi_2^{-1}): E \supset \chi_2(\Omega_{\chi_1} \cap \Omega_{\chi_2})
            \rightarrow \chi_1(\Omega_{\chi_1} \cap \Omega_{chi_2}) \subset E 
        \end{equation*}
        jest dyfeomorfizmem klasy \(p\) (odwzorowaniem \(p\) razy różniczkowalnym w sposób ciągły,
        mającym odwzorowanie odwrotne, również klasy \(p\) razy różniczkowalne w sposób ciągły).
\end{enumerate}
\end{definicja}
\begin{definicja}\cite[s. 253]{maurin2}
    Atlas \(A\) nazywamy \emph{atlasem zupełnym klasy p}, jeśli dodanie do niego mapy powoduje, że
    otrzymany atlas nie będzie już klasy \(p\).
\end{definicja}
\begin{definicja}\cite[s. 253]{maurin2}
    Parę \((M, \{(\chi, \Omega_{\chi})\})\), gdzie \(M\) jest przestrzenią Hausdorffa, a \(\{(\chi, 
        \Omega_\chi)\}\) jest atlasem zupełnym klasy \(p\), nazywa
        się \emph{rozmaitością różniczkowalną klasy p, modelowaną na przestrzeni Banacha \(E\)}
\end{definicja}

\begin{definicja}\cite[s. 300]{maurin2}
    Niech \(W\), \(M\) będą rozmaitościami różniczkowalnymi, \(p: W \rightarrow M\) surjekcją
    (rzutem) różniczkowalną, \(F\) przestrzenią topologiczną, a \(G\)
    efektywną różcznikowalną grupą odwzorowań przestrzeni \(F\) na siebie. Jeżeli
    \begin{enumerate}[label=(\alph*)]
        \item istnieje pokrycie otwarte \((\Omega_i, i\in I)\) przestrzeni \(M\) oraz takie
            dyfeomorfizmy \(h_i: p^{-1}(\Omega_i) \rightarrow \Omega_i \times F\), że
            odwzorywują ,,włókno'' \(p^{-1}(x)\) na \(\{x\} \times F\),
        \item istnieją odwzorowania \(g_{ij}: \Omega_{ij} \rightarrow G\) dla \(i, j \in I\) takie,
            że \(h_i \circ h_j^{-1}(a, f) = (a, g_{ij}(a)f)\) dla \(a\in \Omega_{ij}\), \(f \in F\),
    \end{enumerate}
            to zespół \(W = (W, p, M, F, G)\) nazywamy \emph{przestrzenią wiązki}, \(F\)
            \emph{włóknem typowym},
            \(G\) \emph{grupą struktury wiązki}, \(g_{ij}\) \emph{odwzorowaniem przejścia}, \(h_i\)
            \emph{mapami}, a
            zespół wszystkich tych danych \emph{wiązką różniczkowalną}.
\end{definicja}

\begin{definicja}\cite[s. 17]{evans}
    Niech \(U\) będzie otwartym podzbiorem \(\R^n\), \(k \geq 1\).
    Równanie postaci
    \begin{equation} \label{eq:PDE}
        F(D^k u(x), D^{k-1} u(x), \dots, D u(x), u(x), x) = 0, \, (x\in U)
    \end{equation}
    nazywamy równaniem różniczkowym cząstkowym \(k\)-tego rzędu; funkcja
    \begin{equation*}
        F: \R^{n^k} \times \R^{n^{k-1}} \times \dots \times \R^n \times \R \times U \rightarrow \R
    \end{equation*} jest dana, zaś \(u: U \rightarrow \R\) jest niewiadomą.
\end{definicja}

\begin{definicja}\cite[s. 18]{evans}
    \begin{enumerate}[label=(\roman*)]
        \item Powiemy, że równanie cząstkowe \eqref{eq:PDE} jest liniowe, jeśli jest postaci
            \begin{equation*}
                \sum_{|\alpha |\leq k} a_{\alpha}(x)D^\alpha u = f(x)
            \end{equation*}
        \item  Równanie \eqref{eq:PDE} jest półliniowe lub inaczej semiliniowe, jeśli jest postaci
            \begin{equation*}
                \sum_{|\alpha|=k} a_\alpha (x) D^\alpha u + a_0 (D^{k-1} u, \dots, Du, u, x) = 0
            \end{equation*}
        \item Równanie \eqref{eq:PDE} jest quasi-liniowe, jeśli jest postaci
            \begin{equation*}
                \sum_{|\alpha | = k} a_\alpha (D^{k-1} u, \dots, Du, u, x) D^\alpha u + a_0 (D^{k-1}
                u, \dots, Du, u, x) = 0
            \end{equation*}
        \item Równanie \eqref{eq:PDE} jest całkowicie nieliniowe, jeśli funkcja \(F\) zależy
            nieliniowo od pochodnych najwyższego rzędu.
    \end{enumerate}
\end{definicja}
