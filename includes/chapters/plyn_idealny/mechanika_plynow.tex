\section{Dynamika płynów}

Poniższy opis płynu jest idealizacją, która zaniedbuje cząsteczkową strukturę rzeczywistych płynów. Przez płyny rozumiemy tutaj wszystkie substancje fizyczne w stanie ciekłym i gazowym. \emph{Ciecze} są płynami, które charakteryzuje długość średniej swobodnej drogi cząsteczki w przedziale \(10^{-7} - 10^{-8}\) cm i znaczna odporność na zmianę objętości pod wpływem ciśnienia. \emph{Gazy} charakteryzuje znacznie dłuższa średnia swobodna droga cząsteczek rzędu \(10^{-3}\) cm i łatwość zmiany objętości. Nasze rozważania będą oparte na obserwabliach takich jak prędkość, gęstość i ciśnienie. Wielkości te należy rozumieć jako wartości średnie dla cząsteczek zawartych w infinitezymalnie małej objętości płynu. Przez \emph{cząsteczkę próbną} będziemy rozumieli właśnie taką małą objętość. Nasze rozważania będą konsekwencjami poniższych trzech zasad:
\begin{enumerate}
    \item Masa nie ulega zniszczeniu ani nie jest tworzona.
    \item Zmiana pędu cząsteczki próbnej płynu jest równa wartości przyłożonej do niej siły.
    \item Energia nie ulega zniszczeniu ani nie jest tworzona. 
\end{enumerate}
Drugi punkt to popularna II zasada dynamiki Newtona. Punkt pierwszy wyraża zasadę zachowania masy, zaś trzeci – zasadę zachowania energii. Oznacza to, że wyprowadzając równania ruchu nie uwzględniamy procesów dyssypacji energii, które mogą zachodzić w płynącej cieczy w wyniku wewnętrznego tarcia (\emph{lepkości}) i wymiany ciepła między różnymi częściami cieczy.

Niech \((M, g)\) będzie zwartą, orientowalną \(n\)-rozmaitością Riemannowską z brzegiem i \(\mu \in \Omega^n(M)\) będzie formą objętości na \(M\). Rozmaitość \(M\) interpretujemy jako pewien obszar wypełniony poruszającym się płynem. Ustalmy punkt \(x\in M\) i rozważmy cząstkę próbną, która porusza się po trajektorii przechodzącej przez punkt \(x\) w chwili \(t=0\). Oznaczmy tę trajektorię przez \(\varphi_t(x)=\varphi(x,t)\). Niech \(u(x, t)\) oznacza prędkość cząsteczki próbnej położonej w punkcie \(x\) w chwili \(t\). Określiliśmy w ten sposób na rozmaitości \(M\) pole wektorowe zależne od czasu, które dalej nazywać będziemy \emph{polem prędkości płynu}. Zależność między \(u\) i \(\varphi_t\) możemy wyrazić przez
\begin{equation}\label{eq:first-step}
    \frac{d\varphi_t}{dt}(x) = u(\varphi_t(x), t). 
\end{equation}
Zauważmy, że \(\varphi_t\) jest przepływem dla pola wektorowego zależnego od czasu \(u\). Załóżmy, że dla każdej chwili \(t\) dana jest pewna gładka funkcja \(\rho_t(x)=\rho(x,t)\), która wyraża gęstość płynu w punkcie \(x\). Niech \(W\) będzie podobszarem \(M\). Wówczas masa płynu wypełniającego \(W\) w chwili \(t\) wyraża się przez
\begin{equation}\label{eq:mass}
    m(W,t) = \int_W \rho_t d\mu.
\end{equation}

