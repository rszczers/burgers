\chapter{Płyn idealny} W tym rozdziale naszym celem jest opracowanie modelu fizycznego cieczy idealnej i uzasadnienie jej podstawowych równań ruchu.

\section{Pojęcia wstępne}
Zaczniemy od przypomnienia podstawowych pojęć w teorii rozmaitości gładkich. Niech \(M\) będzie rzeczywistą \(n\)-rozmaitością różniczkowalną klasy \(\mathrm{C}^\infty\), \(p\) będzie dowolnym punktem rozmaitości \(M\) i niech \(U, V\subset M\) będą otoczeniami punktu \(p\). Powiemy, że dwie funkcje rzeczywiste \(f:U\to\R,\ g:V\to\R\) są \emph{równoważne}, jeśli w pewnym otoczeniu \(W\subset U\cap V\) punktu \(p\) są one sobie równe. Pod pojęciem \textbf{kiełka} funkcji rzeczywistej klasy \(\mathrm{C}^\infty\) w punkcie \(p\in M\) będziemy rozumieli każdą klasę równoważności tak zadanej relacji. Zbiór wszystkich kiełków w punkcie \(p\), \(\mathrm{C}_p^\infty(M)\), jest wówczas \(\R\)-algebrą. Zbiór wszystkich funkcji klasy \(\mathrm{C}^\infty\) określonych na \(M\) będziemy oznaczali \(\mathcal{F}(M)\). 

Zmierzamy do określenia pojęcia przestrzeni stycznej za pomocą kiełków funkcji. Każde odwzorowanie liniowe \(D:C_p^\infty(M)\to\R\) spełniające warunek:
\begin{equation}\label{eq:differentiation}
D(fg) = (Df)g(p) + f(Dg)(p)
\end{equation}
nazywamy \textbf{różniczkowaniem} w punkcie \(p\). Przestrzeń liniową wszystkich różniczkowań w punkcie \(p\) nazywamy \textbf{przestrzenią styczną} w \(p\) i oznaczamy przez \(T_pM\).

Niech \(\pi: E\to M\) będzie dowolnym odwzorowaniem między rozmaitościami \(E\) i \(M\). Przeciwobraz \(E_p = \pi^{-1}(p)\) punktu \(p\) nazywamy \textbf{włóknem} w \(p\). Niech \(\pi': E'\to M\) będzie odwzorowaniem rozmaitości \(E'\) i \(M\). Powiemy, że odwzorowanie \(\phi:E\to E'\) \textbf{zachowuje włókna}, jeśli \(\phi(E_p)\subset E'_p\) dla każdego \(p\in M\). Jeśli \(\pi\) jest surjekcją klasy \(\mathrm{C}^\infty\) oraz
\begin{enumerate}[i)]
    \item każde włókno \(\pi^{-1}(p)\) ma strukturę \(r\)-wymiarowej przestrzeni wektorowej,
\item dla każdego \(p\in M\) istnieje otoczenie \(U\in M\) punktu \(p\) i zachowujący włókna dyfeomorfizm \(\phi: \pi^{-1}(U)\to U\times \R^r\) taki, że dla każdego \(q\in U\) zawężenie
\begin{equation*}
\left.\phi\right|_{\pi^{-1}(q)}:\pi^{-1}(q)\to \{q\} \times \R^r
\end{equation*}
jest izomomorfizmem przestrzeni wektorowych,

\end{enumerate}
wówczas \(\pi\) określamy odwzorowaniem \emph{lokalnie trywialnym rzędu r}.  Tak określony zbiór \(U\) nazywamy zbiorem \emph{trywializującym} \(E\), zaś \(\phi\) nazywamy \emph{trywializacją} \(E\) nad \(U\).

Korzystając z powyższych rozstrzygnięć definiujemy \textbf{wiązką wektorową} klasy \(\mathrm{C}^\infty\) rzędu \(r\) jako trójkę \((E, M, \pi)\), gdzie \(\pi: E\to M\) jest lokalnie trywialną gładką surjekcją rzędu \(r\). Nadużywając notacji często zamiast odnosić się bezpośrednio do tak zdefiniowanej struktury, będziemy mówili o \emph{wiązce wektorowej} \(E\) albo \emph{wiązce wektorowej} \(\pi:E\to M\). W tę myśl \textbf{cięciem} wiązki \(\pi: E\to M\) nazywamy odwzorowanie \(s:M\to E\) takie, że \(\pi\circ s= \mathrm{id}_M\). Przyporządkowuje ono każdemu punktowi \(p\in M\) włókno \(E_p\). Jeśli \(E\to M\) jest wiązką wektorową klasy \(\mathrm{C}^\infty\), wówczas przestrzeń wektorową wszystkich cięć \(E\) klasy \(\mathrm{C}^\infty\) oznaczamy \(\Gamma(M, E)\) lub krótko \(\Gamma(E)\), o ile nie prowadzi to do niejednoznaczności. 

Okazuje się, że zbiór wszystkich cięć klasy \(\mathrm{C}^\infty\) wiązki \(E\), \(\Gamma(E)\), ma interesujące własności algebraiczne. Jeśli \(U\) jest otwartym pozdbiorem rozmaitości \(E\), wówczas analogicznie określmy \(\Gamma(U)\). Na tak zadanym zbiorze można zadać strukturę modułu nad pierścieniem \(C^\infty(U)\) funkcji klasy \(\mathrm{C}^\infty\) zadanych na \(U\). \(\Gamma(U)\) jest również rzeczywistą przestrzeńą wektorowa w której możemy wybrać uporządkowaną bazę \((s_1, s_2, \dots, s_r)\). Jest to tak zwany \emph{reper} wiązki wektorowej \(\pi:E\to M\) nad \(U\). 

Szczególnie interesującym nas obiektem tej kategorii jest \textbf{wiązka styczna} \((TM, M, \pi)\), gdzie 
\begin{equation*}
TM := \bigcup\limits_{p\in M} T_p M
\end{equation*}
oraz \(\pi: TM \ni v \mapsto p\in M\) jest naturalną projekcją. 
Jeśli \(M\) jest n-rozmaitością klasy \(\mathrm{C}^k\), to \(TM\) okazuje się \(2n\)-rozmaitością klasy \(\mathrm{C}^{k-1}\) (\cite{tu}, s. 130). 
Korzystając z powyższego wygodnego aparatu matematycznego możemy teraz określić \textbf{pole wektorowe} na rozmaitości \(M\) po prostu jako cięcie wiązki stycznej \(TM\). Zbiór wszystkich pól wektorowych na \(M\), \(\mathfrak{X}(M)\), stanowi szczególny rodzaj \(\mathcal{F}\)-modułu, mianowicie \emph{moduł Liego}. Pola wektorowe można w nim rozumieć jako różniczkowania w algebrze \(\mathcal{F}\) oraz określony jest w nim operator \emph{komutacji} pól wektorowych, nazywany również \emph{nawiasem Liego}:
\begin{equation*}
    [X, Y]f := X(Yf) - Y(Xf).
\end{equation*}
Okazuje się, że komutacja gładkich pól wektorowych nie wyprowadza poza \(\mathfrak{X}(M)\) i lokalnie spełnia warunek \eqref{eq:differentiation}, czyli jest różniczkowaniem.

Niech \(N,\,M\) będą rozmaitościami różniczkowalnymi i \(F:N\to M\) odwzorowaniem klasy \(\mathrm{C}^\infty\) między nimi. Wówczas dla każdego punktu \(p\) na rozmaitości \(M\) określamy odwzorowanie liniowe między przestrzeniami stycznymi 
\begin{equation}
F_{\ast, p}:T_p N\rightarrow T_{F(p)}M, 
\end{equation}
które nazywamy \textbf{różniczką} \(F\) w \(p\in N\). Jeśli \(X_p\in T_p N\) jest wektorem stycznym do \(M\) w \(p\), wówczas dla dowolnego kiełka \(f\in C_{F(p)}^\infty(M)\) funkcji w \(F(p)\) 
\begin{equation}
(F_{\ast, p} (X_p))f = X_p(f\circ F) \in \R.
\end{equation} 
gdzie \(F_{\ast, p} (X_p)\) jest wektorem stycznym do \(N\) w \(F(p)\). Jeśli \(F\) jest dyfeomorfizmem, zaś \(X\) -- polem wektorowym na \(M\), odwzorowanie \(F_\ast \circ X \circ F^{-1}\) nazywamy \textbf{popchnięciem} pola wektorowego \(X\) i przez \(F\). Dualnie, jeśli \(f\in C_p^\infty(M)\) określamy \textbf{cofnięcie} \(F^\ast f := F\circ f\) funkcji \(f\) przez \(F\).


%Przez \(\mathrm{L}^k(V; \R)\) oznaczać będziemy przestrzeń wszystkich funkcjonałów k-liniowych określonych na przestrzeni liniowej \(V\). Rozważmy skończenie wymiarowe rzeczywiste przestrzenie wektorowe \(V_1,\,V_2,\,\dots,\,V_r\) i odpowiednio ich przestrzenie dualne \(V_1^\ast,\,V_2^\ast,\,\dots,\,V_r^\ast\). Dzięki kanonicznemu izomorfizmowi przestrzeni i ich dwusprzężonych, przestrzenie \(V_i\) możemy traktować jako przestrzenie funkcjonałów na \(V_i^\ast\),
%\begin{equation*}
%    x_i: V_i^\ast \ni \varphi^i \mapsto \braket{\varphi^i, x_i}_i := x_i(\varphi^i)\in\R
%\end{equation*}
%gdzie \(\braket{\ast, \ast}_i\) jest funkcją dualności pary przestrzeni \(V_i^\ast,\, V_i\). Możemy określić iloczyn tak rozumianych funkcji,
%\begin{gather*}
%    x_1\otimes \dots \otimes x_r : V_1^\ast\times \dots \times V_r^\ast \mapsto \R,\\
%    (x_1\otimes \dots \otimes x_r)(\varphi^1, \dots, \varphi^r) := \braket{\varphi^1,x_1}_1 \dots \braket{\varphi^r, x_r}_r.
%\end{gather*}
%Tak określone działanie jest łączne, \(r\)-liniowe, ciągłe i nie jest przemienne. Dla wygody oznaczamy \(\mathcal{T}_{0}^{0}(V) = \R\). Wektor postaci \(x_1\otimes\dots \otimes x_r\) nazywamy \emph{iloczynem tenstorowym} wektorów \(x_1, \dots, x_r\).   
Niech \(V\) będzie przestrzenią wektorową. Rozważając funkcjonały wieloliniowe na produkcie kilku kopii \(V\) i jej dualnych otrzymujemy strukturę, którą nazywamy \emph{przestrzenią tensorów o walencji} \({p \brack q}\) nad przestrzenią \(V\),
\begin{align*}
    \mathcal{T}_q^p(V) := \mathrm{L}^{p+q}(\underbrace{V^\ast, \dots, V^\ast}_{p\ \text{razy}} ,\underbrace{V,\dots, V}_{q\ \text{razy}}; \R) , &\quad p+q\geq 1.
\end{align*}
Taką definicję określa się niekiedy \emph{modelem wieloliniowym tensora}. Elementy \(\mathcal{T}_q^p(V)\) nazywamy \emph{tensorami q-krotnie kowariantnymi i p-krotnie kontrawariantnymi} albo prościej: \emph{tensorami o walencji} \(p \brack q\). Dla tensorów \(t_1 \in \mathcal{T}_{s_1}^{r_1}(V)\) i \(t_2 \in \mathcal{T}_{s_2}^{r_2}(V)\) określamy \textbf{iloczyn tensorowy} \(t_1 \otimes t_2 \in \mathcal{T}_{s_1+s_2}^{r_1+r_2}(V)\) przepisem 
\begin{align*}
    (t_1 \otimes t_2)& (\beta^1,\dots, \beta^{r_1}, \gamma^1, \dots, \gamma^{r_2}, f_1, \dots, f_{s_1}, g_1, \dots, g_{s_2}) = \\
                     & t_1(\beta^1,\dots, \beta^{r_1}, f_1, \dots, f_{s_1})\,t_2(\gamma^1, \dots, \gamma^{r_2},g_1, \dots, g_{s_2}),
\end{align*}
gdzie \(\beta^j, \gamma^j \in V^\ast\), \(f_j, g_j\in V\).
Okazuje się (\cite{marsden}, tw. 6.1.2), że jeśli \(\{e_1, \dots, e_n\}\) jest bazą przestrzeni \(V\) i \(\{e^1, \dots, e^n\}\) jest bazą przestrzeni dualnej, to
\begin{equation*}
    \left\{e_{i_1} \otimes \dots \otimes e_{i_q} \otimes e^{j_1} \otimes \dots \otimes e^{j_p} \, | \, i_1, \dots, i_q, j_1, \dots, j_p = 1, \dots, n \right\} 
\end{equation*}
jest bazą \(\mathcal{T}_q^p(V)\), a zatem \(\mathrm{dim}\mathcal{T}_q^p(V) = (\mathrm{dim}V)^{p+q}\). 

Niech \(\pi:\:TM\to M\) będzie wiązką styczną i niech \(T_m M = \pi^{-1}(m)\) oznacza włókno nad punktem \(m\in M\). Określmy
\begin{equation*}
    \mathcal{T}_s^r(M)=\bigcup\limits_{m\in M} \mathcal{T}_r^s(T_m M),
\end{equation*}
Oraz niech odwzorowanie \(\pi_s^r:\mathcal{T}_s^r(M)\to M\), \(\pi_s^r(e) = m\) dla \(e\in \mathcal{T}_s^r(E_m)\) będzie naturalną projekcją elementu z włókna na rozmaitość. Naśladując konstrukcję wiązki stycznej, trójkę \((\mathcal{T}_s^r(M), M, \pi_s^r(e))\) nazywamy \emph{wiązką tensorową} o walencji \(r \brack s\) na \(M\). Analogicznie, cięcie wiązki tensorowej określamy \textbf{polem tensorowym}. 

Okazuje się (\cite{herdegen}, §19, tw. 2), że każda permutacja \(\pi\in S_n\) jednoznacznie wyznacza \emph{operator permutacji wskaźników} \(P^{\pi}:\mathcal{T}_q^p(V)\to\mathcal{T}_q^p(V)\)
\begin{equation*}
    (P^\pi t)(\psi^1, \dots, \psi^p, y_1, \dots, y_q) = t(\psi^{\pi(1)}, \dots, \psi^{\pi(p)}, y_1, \dots, y_q),
\end{equation*} dla \(t\in\mathcal{T}_q^p(V)\). Niech \(\pi\in S_p\) będzie dowolną permutacją zbioru \(p\)-elementowego. Powiemy, że \(t\in\mathcal{T}_q^p(V)\) jest \textbf{tensorem całkowicie symetrycznym (antysymetrycznym) we wskaźnikach kontrawariantnych}, jeśli 
\begin{equation*}
    P^\pi t = t\quad \text{(odpowiednio:}\ P^\pi t = \mathrm{sgn}\pi\, t\text{)}. 
\end{equation*}
Przy powyższych oznaczeniach na przestrzeni \(\mathcal{T}_q^p(V)\) wprowadza się operator
\begin{equation*}
    \mathcal{A}=\frac{1}{p!}\sum_{\pi\in S_p} \mathrm{sgn\,\pi}\,P^\pi,
\end{equation*}
nazywany \emph{operatorem antysymetryzacji}. Można wykazać (\cite{herdegen}, §20, tw. 4), że tensor \(t\) jest antysymetryczny wtedy i tylko, gdy \(\mathcal{A}t = t\). Wprowadźmy oznaczenie\footnote{Jest to nawiązanie do algebraicznej konstrukcji przeprowadzonej za pomocą potęgowania zewnętrznego; okazuje się ona izomorficzna z przyjętymi przestrzeniami; szczegółów można szukać w \cite{maurin2}.} dla przestrzeni tensorów całkowicie antysymetrycznych
\begin{align*}
    \Exterior^p(V):=\mathcal{AT}^p_0(V)\subseteq \mathcal{T}^p_0(V),\quad \Exterior^p(V^\ast):=\mathcal{AT}^0_p(V)\subseteq\mathcal{T}^0_p(V), \quad p\geq 1.
\end{align*}
Ich elementy nazywamy odpowiednio \textbf{\(p\)-wektorami} i \textbf{\(p\)-formami}. Dla wygody przyjmujemy ponadto, że
\begin{equation*}
     \Exterior^0(V):=\Exterior^0(V^\ast)=\R
\end{equation*}
\textbf{Iloczynem zewnętrznych \(p\)-wektorów} (odpowiednio: \textbf{\(p\)-form}) \(t_i\in\Exterior^{p_i}(V)\), \(s^i\in\Exterior^{p_i}(V^\ast)\), \(i=1,\dots, k\), nazywamy odpowiednio tensory
\begin{align*}
    t_1\wedge\dots\wedge t_k = \mathcal{A}(t_1\otimes \dots \otimes t_k), \quad 
    s^1\wedge\dots\wedge s^k = \mathcal{A}(s^1\otimes \dots \otimes s^k).
\end{align*}
Jeśli \(V\) jest przestrzenią \(n\)-wymiarową, wówczas (\cite{herdegen}, §20, tw. 7)
\begin{equation*}
    \mathrm{dim}\,\Exterior^p(V) = \mathrm{dim}\,\Exterior^p(V^\ast) = \binom{n}{k}.
\end{equation*}
W szczególności przestrzenie \(\Exterior^n(V)\) i \(\Exterior^n(V^\ast)\) są jednowymiarowe. 

Będziemy teraz zmierzać do określenia form różniczkowych na rozmaitościach różniczkowalnych. Niech \(p\) będzie punktem z rozmaitości \(M\). \textbf{Przestrzenią kostyczną} w \(p\) do \(M\) nazywamy przestrzeń dualną do przestrzeni stycznej \(T_p M\)
\begin{equation*}
    T^\ast_p M := (T_p M)^\ast = \mathrm{L}(T_p M, \R),
\end{equation*}
Jej elementy są \emph{1-formami}, które nazywamy \textbf{kowektorami}. Podobnie jak w przypadku wiązki stycznej, określając
\begin{equation*}
    T^\ast M = \bigcup\limits_{p\in M} T_p^\ast M.
\end{equation*}
możemy zadać strukturę wiązki wektorowej \((T^\ast M, M, \phi)\) rzędu \(2n\) (\cite{tu}, s. 192), którą nazywamy \textbf{wiązką kostyczną} na \(M\). 
Rozważmy zbiór
\begin{equation*}
    \Exterior^k(T^\ast M) := \bigcup\limits_{p\in M} \Exterior^k(T^{\ast}_p M) = \bigcup\limits_{p\in M} \mathcal{AT}^0_k(T_p M)
\end{equation*} i zadajmy odwzorowanie \(\pi: \Exterior^k (T^\ast M) \ni \alpha \mapsto p\in M\) dla 
\(\alpha \in \Exterior^k(T^{\ast}_p M)\). Okazuje się (\cite{tu}, s. 203), że \(\pi\) jest wiązką wektorową rzędu \(\binom{n}{k}\), której cięcia są antysymetrycznymi \(k\)-kowariantnymi (i \(0\)-kontrawariantnymi) polami tensorowymi na \(M\).   
Każde \(k\)-kowariantne pole tensorowe na \(M\), które jest antysymetryczne, nazywamy \textbf{\(k\)-formą różniczkową} na \(M\).
Przestrzeń wektorową wszystkich \(\mathrm{C}^\infty\) \(k\)-form na \(M\) oznaczamy \(\Omega^k(M)\) oraz
\begin{equation*}
    \Omega^k(M)=\Gamma\left(M, \Exterior^k(T^\ast M)\right).
\end{equation*}


%\begin{twierdzenie}[\cite{herdegen}, §19, tw. 2]
%    Dla każdej permutacji \(\pi\in S_p\) i istnieje dokładnie jeden operator liniowy \(P^{\pi}:\mathcal{T}_q^p(V)\to\mathcal{T}_q^p(V)\), który na tensory proste działa według przepisu
%    \begin{equation*}
%        P^\pi(x_1\otimes \dots \otimes x_p \otimes \varphi^1 \otimes \dots \otimes \varphi^q) = x_{\pi^{-1}(1)} \otimes \dots \otimes x_{\pi^{-1}(p)}\otimes \varphi^1 \otimes \dots \otimes \varphi^q
%    \end{equation*}
%\end{twierdzenie}
%Każde kowariantne (\(0\)-kontrawariantne) pole tensorowe na \(M\), które jest antysymetryczne, nazywamy \textbf{k-formą różniczkową} na \(M\).

%Jeśli \(V\) jest przestrzenią wektorową, przez \(A_k(V)\subset \mathcal{T}_k^0(V)\) oznaczamy podprzestrzeń liniową alternujących k-tensorów na \(V\), oznaczaną czasem \(\Exterior^k(V^{\vee})\) przez nawiązanie do izomorficznej konstrukcji na gruncie algebr Grassmanna. W tej pracy wyłącznie zaznaczymy tę odpowiedniość przez stosowanie tej notacji, gdyż wzmiankowana konstrukcja daleko wykracza poza jej zakres. Piszemy odpowiednio:
%\begin{align*}
%    \Exterior^0(V^{\vee}) &= A_0(V) = \R\\
%    \Exterior^1(V^{\vee}) &= A_1(V) = V^\ast\\
%    \Exterior^2(V^{\vee}) &= A_2(V),\ \text{i tak dalej.}
%\end{align*}
%Nasze największe zainteresowanie będą budziły gładkie k-formy różniczkowe. Przestrzeń liniową wszystkich k-form na rozmaitości \(M\) klasy \(\mathrm{C}^\infty\) konwencjonalnie oznaczamy przez \(\Omega^k(M)\). 
%Rozważmy przestrzenie wektorowe \(V_1, V_2, \dots, V_r\). Przestrzeń \(V\) nazwiemy \emph{iloczynem tensorowym} przestrzeni \(V_1, V_2, \dots, V_r\) i będziemy oznaczali \(V=V_1\otimes V_2 \otimes \dots \otimes V_r\), jeśli 
%\begin{enumerate}[i)]
%    \item zadane jest odwzorowanie wieloliniowe
%        \begin{equation*}
%            \otimes: V_1\times \dots \times V_r \ni (x_1, \dots, x_r) \mapsto x_1\otimes\dots \otimes x_r\in V,\\
%        \end{equation*}
%    \item zbiór wektorów \(\left{e_1\otimes \dots e_r | e_{alpha} 
%
%\end{enumerate}

%Mnożąc tensorowo kilka kopii jednej przestrzeni wektorowej i jej dualnej otrzymujemy kolejną istotną strukturę. \textbf{Przestrzenią tensorów o walencji} \({p \brack q}\) nad przestrzenią \(V\) nazwiemy przestrzeń wektorową
%\begin{align*}
%    \mathcal{T}_q^p(V) := \mathrm{L}^{p+q}(\underbrace{V^\ast \otimes \dots \otimes V^\ast}_{p\ \text{razy}} \otimes\underbrace{V\otimes\dots\otimes V}_{q\ \text{razy}}; \R) , &\quad p+q\geq 1.
%\end{align*}
%Elementy \(\mathcal{T}_q^p(V)\) nazywamy \emph{tensorami q-krotnie kowariantnymi i p-krotnie kontrawariantnymi} albo prościej: \emph{tensorami o walencji} \(p \brack q\). Okazuje się, że jeśli \(\{e_1, \dots, e_n\}\) jest bazą przestrzeni \(V\) i \(\{e^1, \dots, e^n\}\) jest bazą przestrzeni dualnej, to
%\begin{equation*}
%    \left\{e_{i_1} \otimes \dots \otimes e_{i_q} \otimes e^{j_1} \otimes \dots \otimes e^{j_p} \, | \, i_1, \dots, i_q, j_1, \dots, j_p = 1, \dots, n \right\} 
%\end{equation*}
%jest bazą \(\mathcal{T}_q^p(V)\), a zatem \(\mathrm{dim}\mathcal{T}_q^p(V) = (\mathrm{dim}V)^{p+q}\).

%Wybierając w otoczeniu punktu \(a\) na rozmaitości \(M\) lokalny układ współrzędnych \((U, \varphi)=(U, x_1, \dots, x_r)\), a więc wybierając w \(T_a M\) reper \(\left(\left.\frac{\partial}{\partial x^1}\right|_a, \dots, \left.\frac{\partial}{\partial x^r}\right|_a\right)\) wybieramy tym samym w przestrzeni \(\mathcal{T}_q^p(a):= \mathcal{T}_q^p(T_aM)\) bazę iloczynową
%\begin{equation*}
%    \left\{\frac{\partial}{\partial x^{i_1}} \otimes \dots \otimes \frac{\partial}{\partial x^{i_r}} \otimes dx^{j_1} \otimes \dots \otimes dx^{j_s}\right\}.
%\end{equation*}
%
%\textbf{Polem tensorowym} o walecji \(p \brack q\) klasy \(\mathrm{C}^k,\ k\leq r\), na rozmaitości różniczkowalnej klasy \(\mathrm{C}^r\) nazywamy odwzorowanie 
%\begin{gather*}
%    w: M \mapsto \mathcal{T}_q^p (M) := \bigcup\limits_{x\in M}\mathcal{T}_q^p (x),  \quad w(x) \in \mathcal{T}_q^p (x),
%\end{gather*}
%gdzie składowe tensora \(w(x)\) są funkcjami klasy \(\mathrm{C}^k(M)\). Przy powyższych określeniach \textbf{formą różniczkową} rzędu \(p\) na \(M\) będziemy nazywali każde antysymetryczne pole tensorowe \(p\)-kowariantne i \(0\)-kontrawariantne.
Niech \(\omega\) będzie \(k\)-formą różniczkową i niech \(X_1, \dots, X_k\) będą polami wektorowymi na \(M\). Określamy na \(M\) funkcję \(\omega(X_1, \dots, X_k)\) postaci
\begin{equation*}
    M\ni p \mapsto \omega(X_1, \dots, X_k)_p = \omega_p\left((X_1)_p, \dots, (X_k)_p\right) \in \R,\quad p\in M
\end{equation*}
Niech \(f\in C^\infty(M)\) będzie funkcją na rozmaitości \(M\) i \(X\) będzie polem wektorowym na \(M\). Określmy \(1\)-formę różniczkową \(df\) 

\textbf{Różniczką} funkcji \(f\) nazywamy \(1\)-formę różniczkową \(df\) na \(M\) taką, że dla każdego \(p\in M\) i wektora stycznego \(X_p\in T_p M\) zachodzi
\begin{equation*}
    \left.df\right|_p(X_p) = X_p f.
\end{equation*}

Niech \(X\) będzie polem wektorowym klasy \(\mathrm{C}^r\) na \(M\) i niech \((U, \varphi) = (U, x^1, x^2, \dots, x^n)\) będzie mapą wokół \(p\in M\). Wówczas \(X_p = \sum_{j=1}^{n}a_j(p)\left.\frac{\partial}{\partial x^j}\right|_p\) jest wektorem stycznym w p, gdzie \(a_j\in C^r(M)\) są kiełkami funkcji klasy \(\mathrm{C}^r\) w \(p\). Popchnięcie \(X\) przez \(\varphi\), czyli funkcję wektorową \(\mathbf{X} = \varphi_{\ast}\circ X(\varphi^{-1}): \R^n \supset \varphi(U)\ni p \mapsto \left[a_j(p)\right]_{j=1}^n\in \R^n\) nazywamy \textbf{lokalną reprezentacją} X.

\begin{figure}[!h]
\centering
\begin{tikzpicture}
  \matrix (m) [matrix of math nodes,row sep=3em,column sep=4em,minimum width=2em]
  {
     M\supset U & \varphi(U)\subset \R^n \\
     TU & T\R^n \cong \R^n \\};
  \path[-stealth]
    (m-1-1) edge node [left] {\(X\)} (m-2-1)
    (m-1-1) edge node [above] {\(\varphi\)} (m-1-2)
    (m-1-2) edge node [right] {\(\mathbf{X}\)} (m-2-2)
    (m-2-1) edge node [above] {\(\varphi_{\ast}\)} (m-2-2);
\end{tikzpicture}
\end{figure}

Chwilą \(t\in\R\) będziemy nazywać zmienną czasową. \textbf{Polem wektorowym zależnym od czasu} klasy \(\mathrm{C}^r\) na \(M\) nazywamy odwzorowanie \(X:\R\times M\rightarrow TM\) takie, że \(X_t(m):=X(t,\,m) \in T_{m} M\) jest wektorem stycznym w \(m\) w chwili \(t\) dla wszytkich par \((t, m) \in \R \times M\). Przez \(X_t\in \mathfrak{X}^r(M)\) oznaczamy pole wektorowe na \(M\) w chwili \(t\), gdzie \(\mathfrak{X}^r(M)\) to zbiór wszystkich pól wektorowych klasy \(\mathrm{C}^r\) na \(M\). 

\textbf{Trajektorią} (także: linią przepływu, krzywą całkową) pola wektorowego \(X\) w punkcie \(m\in M\) nazywamy krzywą \(c: \R \supset I \rightarrow M\) o początku w \(m\), taką, że \(c'(t) = X_{c(t)}\) dla każdego \(t\in I\). Jeśli \((U, \varphi) = (U, x^1, x^2, \dots, x^n)\) jest mapą wokół \(c(0)=p\) i \([X^1, X^2, \dots, X^n]^T\) jest lokalną reprezentacją \(X\), funkcja wektorowa \(\mathbf{c} = \varphi \circ c, I \ni t \mapsto \left[c^i(t)\right]_{i=1}^m \in \R^n\) jest lokalną reprezentacją krzywej \(c\) oraz spełniony jest układ równań różniczkowch pierwszego rzędu nazywany układem charakterystyk
\begin{align*}
    \frac{dc}{dt}^1(t) &= X^1\left(c^1(t), c^2(t), \dots, c^n(t)\right),\\
    \frac{dc}{dt}^2(t) &= X^2\left(c^1(t), c^2(t), \dots, c^n(t)\right),\\
    &\mathrel{\makebox[\widthof{=}]{\vdots}}  \\
    \frac{dc}{dt}^n(t) &= X^n\left(c^1(t), c^2(t), \dots, c^n(t)\right).
\end{align*}
\textbf{Prędkością} \(c'(t_0)\) \textbf{krzywej} \(c\) w chwili \(t\in]a, b[\) nazywamy wektor styczny
\begin{equation}
c'(t_0) = c_\ast\left(\left.\frac{d}{dt}\right|_{t_0}\right)\in T_{c(t_0)}M,
\end{equation}
Zachodzą następujące twierdzenia
\begin{twierdzenie}
Niech \(X_p\) będzie wektorem stycznym w punkcie \(p\) rozmaitości \(M\) i niech \(f\in C_p^\infty(M)\) będzie kiełkiem funkcji \(C^\infty\) w \(p\). Jeśli \(c:\:]-\varepsilon,\varepsilon[\rightarrow M\) jest gładką krzywą o początku w \(p\) taką, że \(c'(0)=X_p\), wówczas
\begin{equation}
X_pf=\left.\frac{d}{dt}\right|_0(f\circ c).
\end{equation}
\end{twierdzenie}
\begin{twierdzenie}
Niech \(F: N\to M\) będzie gładkim odwzorowaniem między rozmaitościami, \(p\in N\), \(X_p\in T_p N\). Jeśli \(c\) jest gładką krzywą o początku w \(p\) i jej prędkość w \(p\) to \(X_p\), wówczas
\begin{equation}
F_{\ast, p} (X_p) = \left.\frac{d}{dt}\right|_0(F\circ c)(t).
\end{equation}
Czyli popchnięcie prędkości \(X_p\) przez \(F\) jest wektorem prędkości krzywej \(F\circ c\) w \(M\).
\end{twierdzenie}

Przez \(\mathcal{D}_X\) oznaczmy wszystkie pary \((m, t)\in M \times \R\) dla których istnieje trajektoria \(c: I\rightarrow M\) pola wektorowego \(X\) w punkcie \(m\) i zmienna czasowa \(t\) zawiera się w pewnym przedziale \(I\). Mówimy, że pole wektorowe jest \textbf{zupełne}, jeśli \(\mathcal{D}_X = M \times \R\). Oznacza to, że dla każdego punktu na rozmaitości znajdziemy trajektorię cząsteczki próbnej poruszającej się dowolnie długo. Zbiór wszystkich punktów rozmaitości na których pole wektorowe nie znika, \(\mathrm{Supp}\,X=\{m\in M\:|\: X_m\neq 0\}\), nazywamy \textbf{nośnikiem} pola wektorowego X. 

Zachodzą następujące twierdzenia

\begin{twierdzenie}\label{tw:D}
Niech \(X\) będzie polem wektorowym na \(M\) klasy \(C^r,\, r\geq1\). Wówczas
\begin{enumerate}[i)]
\item \(\mathcal{D}_X\supset M\times \{0\}\),
\item \(\mathcal{D}_X\) jest otwarty w \(M\times \R\),
\item istnieje jednoznacznie wyznaczone odwzorowanie \(F_X:\mathcal{D}_X\rightarrow M\) takie, że krzywa \(t\mapsto F_X(m,t)\) jest trajektorią w \(m\) dla wszystkich \(m\in M\),
\item dla \((m,t)\in \mathcal{D}_X,\) \(F_X(m,t),\, s)\in\mathcal{D}_X\) wtedy i tylko wtedy, gdy \((m, t+s)\in \mathcal{D}_X\).
\end{enumerate}
\end{twierdzenie}  

\begin{twierdzenie}
Każde pole wektorowe klasy \(C^r\) na zwartej rozmaitości \(M\) jest zupełne.
\end{twierdzenie}

Określone w Twierdzeniu \ref{tw:D} odwzorowanie \(F_X\) nazywamy \textbf{całką} \(X\), zaś trajektorię \(t\rightarrow F_X(m,t)\) \textbf{maksymalną krzywą całkową} \(X\) w \(m\). Jeśli pole wektorowe \(X\) jest zupełne, \(F_X\) nazywamy \textbf{przepływem} pola wektorowego \(X\). Każdy przepływ \(F\) określa 1-parametrową grupę dyfeomorfizmów \(\{F_t:\:M\rightarrow M\:|\: t\in \R\}\) z operacją składania \(F_{t_1}\circ F_{t_2} = F_{t_1 + t_2}\) dla \(t_{1}, {t}_2 \in \R\), gdzie \(F_0\) jest elementem neutralnym i \(F_{t}\circ F_{-t} = F_{0}\) dla dowolnego \(t\in\R\). Jeśli \(X\) jest polem wektorowym zależnym od czasu, wówczas analogicznie określamy \textbf{przepływ zależny od czasu} \(F_{t,s}\) pola \(X\) dla którego odwzorowanie \(t\mapsto F_{t,s}(m)\) jest trajektorią \(X\) o początku w punkcie \(m\) i w chwili \(t=s\), czyli
\begin{align}
\frac{d}{dt}F_{t,s}(m) = X\left(t, F_{t,s}(m)\right), &\quad F_{s,s}(m) = m.
\end{align} 
Wówczas działanie składania staje się przechodnie, \(F_{t,s}\circ F_{s,r} = F_{s, r}\), a \(F_{t, t}\) jest jego elementem neutralnym. 

%Pojęcia pola wektorowego, przepływu i trajektorii wiąże następujące twierdzenie
%
%\begin{twierdzenie}[O lokalnym istnieniu gładkiego przepływu]
%Niech \((U, \varphi)\) będzie mapą wokół \(p\in M\) i niech \(X\) będzie polem wektorowym klasy \(C^\infty\) na \(U\). Wówczas dla każdego \(p_0\in U\) istnieje przepływ \(F:]-\varepsilon, \varepsilon[ \times V \rightarrow U\) klasy \(C^\infty\) dla pewnego otoczenia \(V\subset U\) i pewnego \(\varepsilon>0\) taki, że \(F_t\) jest trajektorią pola wektorowego \(X\) w \(p_0\), czyli
%\begin{align}
%\frac{\partial F}{\partial t}(t, q) = X_{F_t(q)}, &\quad F_0(p_0) = p_0,  
%\end{align}
%dla wszystkich \((t, q) \in ]-\varepsilon, \varepsilon[ \times V\)
%\end{twierdzenie}
%\begin{twierdzenie}[O lokalnym istnieniu gładkiego i jednoznacznego przepływu]
%    Niech \(U\) będzie otwartym podzbiorem \(\R^n\) i niech \(\mathbf{X}: U\times \R \rightarrow \R^n\) będzie lokalną reprezentacją pola wektorowego zależnego od czasu klasy \(C^r\,, r\geq 1\). Wówczas 
%    \begin{enumerate}[i)]
%        \item Dla dowolnego \(x_0\in U\) i chwili \(t\in \R\) istnieje trajektoria w \(x_0\),
%    \item Jeśli w punkcie i w tej samej chwili istnieją dwie różne trajektorie, to są identyczne na przecięciu swoich dziedzin,
%    \item Istnieje otoczenie \(U_0\) punktu \(p\in U\) oraz przepływ \(F: U_0 \times ]-a, a[ \rightarrow \R^n\) klasy \(C^r\) dla pewnego \(a>0\) takie, że krzywa \(c_u: \left]-a, a\right[\rightarrow \R^n\, c_u(t) = F(u, t)\) jest trajektorią w \(u\in \R^n\).
%\end{enumerate}
%\end{twierdzenie}
%Wówczas \(u(x, t)\) oznacza prędkość cząsteczki próbnej przechodzącej przez punkt \(x\in M\). 

Niech \(X,\,Y\in\mathfrak{X}(M)\) i \(F:]-\varepsilon, \varepsilon[\times U\rightarrow M\) będzie lokalnym przepływem pola wektorowego \(X\) w otoczeniu \(U\subset M\) punktu \(p\in M\). \textbf{Pochodną Liego} \(\mathcal{L}_X Y\) \textbf{pola wektorowego} \(Y\) względem \(X\) w \(p\) nazywamy wektor
\begin{equation}\label{eq:lie_vector}
\begin{split}
(\mathcal{L}_X Y)_p &= \lim_{t\to 0} \frac{F_{-t\ast}(Y_{F_t(p)})-Y_p}{t} =\\
                    &= \lim_{t\to 0} \frac{(F_{-t\ast}Y)_p - Y_p}{t} =\\
                    &= \left.\frac{d}{dt}\right|_{t = 0}(F_{-t\ast}Y)_p.
\end{split}
\end{equation} 
Jeśli \(\omega\) jest gładką k-formą na rozmaitości \(M\), to \textbf{pochodną Liego} \(\mathcal{L}_X \omega\) \textbf{k-formy} \(\omega\) względem \(X\) w \(p\in M\) nazywamy formę
\begin{equation}\label{eq:lie_form}
\begin{split}
(\mathcal{L}_X \omega)_p &= \lim_{t\to 0}\frac{F_t^{\ast}(\omega_{F_t(p)})-\omega_p}{t} =\\
						 &= \lim_{t\to 0}\frac{(F_t^{\ast}\omega)_p - \omega_p}{t} =\\
						 &= \left.\frac{d}{dt}\right|_{t=0}(F_t^{\ast}\omega)_p.
\end{split}
\end{equation}

Niech \((M, g)\) będzie zwartą, orientowalną \(n\)-rozmaitością Riemannowską z brzegiem i \(\mu \in \Omega^n(M)\) będzie formą objętości na \(M\). Przypomnijmy, że metryka Riemannowska \(g:\: M \rightarrow \Omega^2(M)\) to pole tensorowe \(\mathcal{T}_2^0(M)\) takie, że \(g_p \in T_m^{\ast} M \otimes T_p^{\ast} M\) dla \(p\in M\) jest iloczynem skalarnym określonym na przestrzeni stycznej \(T_p M\). \textbf{Formą objętości} na n-rozmaitości \(M\) nazywamy n-formę \(\mu\in\Omega^n(M)\) taką, że \(\mu(m)\neq 0\) dla wszystkich \(m\in M\). Mówimy, że \(M\) jest orientowalna, jeśli na \(M\) można określić formę objętości. 

Niech \(X\) będzie polem wektorowym na \(M\). Funkcję \(\mathrm{div}_\mu X \in C^\infty(M)\) taką, że 
\begin{equation}
\mathcal{L}_X \mu = (\mathrm{div}_\mu X) \mu
\end{equation}
nazywamy \textbf{dywergencją} \(X\). Mówimy, że \(X\) jest \textbf{nieściśliwy} (względem \(\mu\)), jeśli \(\mathrm{div}_\mu X = 0\).
