\chapter{Płyn idealny} Naszym celem jest uzasadnienie podstawowych równań ruchu cieczy idealnej. 

Niech \((M, g)\) będzie zwartą, orientowalną \(n\)-rozmaitością Riemannowską z brzegiem i \(\omega \in \Omega^n(M)\) będzie formą objętości na \(M\). Przypomnijmy, że \(g\) jest indeksowaną przez \(M\) rodziną iloczynów skalarnych określonych na przestrzeniach stycznych do M, \(g_p: T_p M \times T_p M \rightarrow \R\) dla \(p\in M\) i \(p\mapsto g_p(X_p, Y_p)\), gdzie \(X\) i \(Y\) są różniczkowalnymi polami wektorowymi na \(M\).

Niech \(X\) będzie polem wektorowym klasy \(C^r\) na \(M\) i niech \((U, \varphi) = (U, x^1, x^2, \dots, x^n)\) będzie mapą wokół \(p\in M\). Wówczas \(X_p = \sum_{j=1}^{n}a_j(p)\left.\frac{\partial}{\partial x^j}\right|_p\) jest wektorem stycznym w p, gdzie \(a_j\in C^r(M; \R)\). Funkcję wektorową \(\mathbf{X}: U\ni p \mapsto \left[a_j(p)\right]_{j=1}^n\in \R^n\) nazywamy \textbf{lokalną reprezentacją} X.

Przez chwilę \(t\in\R\) będziemy na ogół oznaczać zmienną czasową. \textbf{Polem wektorowym zależnym od czasu} klasy \(C^r\) na \(M\) nazywamy odwzorowanie \(X:\R\times M\rightarrow TM\) takie, że \(X_t(m):=X(t,\,m) \in T_{m} M\) jest wektorem stycznym w \(m\) w chwili \(t\) dla wszytkich par \((t, m) \in \R \times M\). Przez \(X_t\in \mathfrak{X}^r(M)\) oznaczamy pole wektorowe na \(M\) w chwili \(t\), gdzie \(\mathfrak{X}^r(M)\) to zbiór wszystkich pól wektorowych klasy \(C^r\) na \(M\). 

\textbf{Przepływem} (także operatorem ewolucji) na \(M\) nazywamy 1-parametrową grupę dyfeomorfizmów \(\varphi_t: M \rightarrow M\) z operacją składania \(\varphi_{t_1}\circ\varphi_{t_2} = \varphi_{t_1 + t_2}\) dla \(t_{1}, {t}_2 \in \R\), gdzie \(\varphi_0\) jest elementem neutralnym i \(\varphi_{t}\circ\varphi_{-t} = \varphi_0\) dla dowolnego \(t\in\R\).  

\textbf{Trajektorią} (także: linią przepływu, krzywą całkową) pola wektorowego \(X\) w punkcie \(m\in M\) nazywamy krzywą \(c: \R \supset I \rightarrow M\) o początku w \(m\), taką, że \(c'(t) = X(c(t))\) dla każdego \(t\in I\). Jeśli \((U, \varphi) = (U, x^1, x^2, \dots, x^n)\) jest mapą wokół \(c(0)=p\) i \([X^1, X^2, \dots, X^n]^T\) jest lokalną reprezentacją X, funkcja wektorowa \(\mathbf{c} = \varphi \circ c, I \ni t \mapsto \left[c^i(t)\right]_{i=1}^m \in \R^n\) jest lokalną reprezentacją krzywej \(c\) oraz spełniony jest układ równań różniczkowch pierwszego rzędu nazywany układem charakterystyk
\begin{align*}
    \frac{dc}{dt}^1(t) &= X^1(c^1(t), c^2(t), \dots, c^n(t)),\\
    \frac{dc}{dt}^2(t) &= X^2(c^1(t), c^2(t), \dots, c^n(t)),\\
    &\mathrel{\makebox[\widthof{=}]{\vdots}}  \\
    \frac{dc}{dt}^n(t) &= X^n(c^1(t), c^2(t), \dots, c^n(t)).
\end{align*}

Pojęcia pola wektorowego, przepływu i trajektorii wiąże następujące twierdzenie

\begin{twierdzenie}[O lokalnym istnieniu gładkiej i jednoznacznej trajektorii]
    Niech \(U\) będzie otwartym podzbiorem \(\R^n\) i niech \(\mathbf{X}: U\times \R \rightarrow \R^n\) będzie lokalną reprezentacją pola wektorowego zależnego od czasu klasy \(C^r\, r\geq 1\). Wówczas 
    \begin{enumerate}[i)]
        \item Dla dowolnego \(x_0\in U\) i chwili \(t\in \R\) istnieje trajektoria w \(x_0\),
    \item Jeśli w punkcie i w tej samej chwili istnieją dwie różne trajektorie, to są identyczne na przecięciu swoich dziedzin,
    \item Istnieje otoczenie \(U_0\) punktu \(p\in U\) oraz przepływ \(F: U_0 \times ]-a, a[ \rightarrow \R^n\) klasy \(C^r\) dla pewnego \(a>0\) takie, że krzywa \(c_u: \left]-a, a\right[\rightarrow \R^n\, c_u(t) = F(u, t)\) jest trajektorią w \(u\in \R^n\).
\end{enumerate}
\end{twierdzenie}
Wówczas \(u(x, t)\) oznacza prędkość cząsteczki próbnej przechodzącej przez punkt \(x\in M\). 
