Naszym celem jest uzasadnienie podstawowych równań ruchu cieczy idealnej. 

\begin{definicja}
Polem wektorowym zależnym od czasu klasy \(\mathcal{C}^r\) na rozmaitości \(M\) nazywamy odwzorowanie \(X: \mathbb{R} \times M \rightarrow TM\) takie, że \(X(t,\,m) \in T_{m} M\) dla wszytkich par \((t, m) \in \R \times M\).
\end{definicja}

Niech \(M\) będzie orientowalną \(n\)-rozmaitością Riemannowską z gładkim brzegiem
\(u:\, \) - .  

Wówczas \(u(x, t)\) oznacza prędkość cząsteczki próbnej przechodzącej przez punkt \(x\in M\). 
